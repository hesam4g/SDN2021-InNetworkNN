\section{Introduction}
Reconfigurable network cards- or in other words \smartnics, offering more resources in datacenters, are becoming increasingly popular in recent years. There are several kinds of \smartnic, and in this report, first, we will briefly introduce two main architectures of them and their advantages. Afterward, we will review Artificial Neural Networks (ANN) to provide adequate background for our work. This work will focus on executing Machine Learning (ML) and Deep Learning (ML) applications on \smartnics. Our primary reference \cite{siracusano2020running} has examined ML/DL Neural Networks (NNs) on two types of \smartnics, but they penalized some parameters (e.g., the accuracy of NNs); however, the throughput and latency are enhanced significantly.
\par
By and large, each \smartnic has its limitations and advantages. We aim to figure out whether we can execute these applications on a \smartnic without sacrificing the accuracy. So, we will pick the \smartnic which has not been investigated for these purposes and investigate if the \smartnic helps us to achieve our goal or not. Our results show the selecting the suitable model for our \smartnic's type is crucial due to its limitation. We will evaluate two types of ANNs, and yield some information showing which one can be a better fit than the other.

\par
With this in mind, the rest of the article is organized as the following. In section \ref{sec-back}, \smartnics are introduced. We will also talk about artificial neural networks in section \ref{sec-back}. In \ref{sec-motive} we talk about offloading ANN into \smartnics and explain our goal. Our testbed details and results are studied in sections \ref{sec-testbed} and \ref{sec-results}, respectively. Then, we will address a few related works in section \ref{sec-related}. Eventually, in section \ref{sec-conlcu}, we will conclude our project based on the provided results.